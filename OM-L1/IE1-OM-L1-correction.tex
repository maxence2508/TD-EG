\documentclass[a4paper,12pt]{article}

% Packages
\usepackage[utf8]{inputenc}
\usepackage[T1]{fontenc}
\usepackage[french]{babel}
\usepackage{amsmath,amsfonts,amssymb}
\usepackage{graphicx}
\usepackage{geometry}
\usepackage{fancyhdr}
\usepackage{lastpage}
\usepackage{lipsum} % Pour générer du faux texte. À retirer dans le document final.
\usepackage{natbib}
\usepackage{bbm}
\usepackage{titlesec}
\usepackage{tcolorbox}

% Paramètres de la page
\geometry{top=2cm, bottom=2cm, left=2.5cm, right=2.5cm, footskip=1cm}
\pagestyle{fancy}
\fancyhf{}
\lhead{Outils Mathématiques : Interrogation 1}
\cfoot{\thepage} % Numéro de page au milieu du pied de page
\renewcommand{\headrulewidth}{2pt}
\renewcommand{\footrulewidth}{0pt} % Pas de ligne en bas
\setlength{\parindent}{0pt}

% Mise en page de la première page
\fancypagestyle{plain}{
  \fancyhf{} % Supprime les en-têtes et pieds par défaut
  \cfoot{\thepage} % Numéro de page au milieu
  \renewcommand{\headrulewidth}{0pt} % Pas de ligne d'en-tête
  \renewcommand{\footrulewidth}{0pt} % Pas de ligne en bas
}

% Titre du document
\title{Outils Mathématiques \\
Année 2024-2025}
\author{Interrogation 1}
\date{6 Mars 2025 \\
Durée : 1h}

% Macros
\newcommand{\E}{\mathbb{E}}
\newcommand{\prob}{\mathbb{P}}
\newcommand{\R}{\mathfrak{R}_0}
\newcommand{\Hun}{H^1(\Omega)}
\newcommand{\Ho}{H^1_0(\Omega)}
\newcommand{\vois}{\mathring{\mathcal{U}}}
\newcommand{\ind}{\mathbbm{1}}
\newcommand{\f}{\mathcal{F}}
\newcommand{\n}{\mathbb{N}}
\newcommand{\m}{\mathcal{M}}
\newcommand{\norm}{\mathcal{N}}
\newcommand{\sme}{\mathcal{E}}

% Chargement de hyperref avec les options personnalisées pour les cadres rouges
\usepackage[
    colorlinks=false,         % Pas de lien coloré
    linkbordercolor={1 0 0},  % Cadre rouge autour des liens
    pdfborder={0 0 1}         % Taille du cadre autour des liens
]{hyperref}

% Supprimer le décalage des sous-sections et masquer le numéro
\titleformat{\subsection}[hang]{\normalfont\bfseries}{}{0pt}{}

\begin{document}

\maketitle

\textbf{Attention : Lorsqu'un calcul est demandé, il est attendu que les étapes permettant d'aboutir au résultat soient détaillées. Plus généralement, toute réponse doit être justifiée.} \\


\textbf{Exercice 1} (Inversion matricielle) \\

\textbf{1/} Soient \( A = \begin{pmatrix}
2 & -2 \\
1 & -2 \\
\end{pmatrix} \) et \( B = \begin{pmatrix}
1 & 2 & -2 \\
0 & 1 & -2 \\
1 & 1 & 1
\end{pmatrix} \). \\

Inverser les matrices $A$ et $B$ en appliquant la méthode du pivot. \\

\textbf{2/} Soit \( C = \begin{pmatrix}
0 & 1 & -1 \\
-1 & 2 & -1 \\
1 & -1 & 2
\end{pmatrix} \).  \\

\textbf{a)} Calculer $C^2$ puis $C^2-3C$. \\

\textbf{b)} En déduire que $C$ est inversible et donner l'expression de son inverse $C^{-1}$. \\

\textbf{Exercice 2} (Diagonalisation) \\

Soit \( A = \begin{pmatrix}
3 & 0 & 0 \\
2 & 2 & 0 \\
4 & 0 & 1
\end{pmatrix} \).\\

\textbf{1/} Calculer le polynôme caractéristique de la matrice $A$ en le laissant sous forme factorisée. En déduire l'ensemble des valeurs propres de la matrice $A$. \\

\textbf{2/} Trouver une base de chacun des sous-espaces propres de la matrice $A$. \\

\textbf{3/} La matrice $A$ est-elle diagonalisable ? Justifier. \\

\vfill
\begin{flushright}
\textbf{Tourner la page S.V.P.}
\end{flushright}

\textbf{Exercice 3} (Loi binomiale) \\

\begin{tcolorbox}[colframe=black, colback=white, sharp corners, width=\textwidth, boxrule=0.5mm]
\textbf{Rappel :}\\
\begin{itemize}
    \item On dit que la variable aléatoire $X$ suit la \textbf{loi binomiale} de paramètres $n \in \n^*$, $p \in \ ]0,1[$ et on note $X \sim \mathcal{B} (n,p)$ si $X$ prend ses valeurs dans $\{0,1,...,n\}$ et si pour tout $k \in \{0,1,...,n\}$, $\prob (X=k) = \binom{n}{k} p^k (1-p)^{n-k}$. \\

\item On rappelle la \textbf{formule du capitaine}. Pour $1 \leq k \leq n$, on a :
$$k \binom{n}{k} = n \binom{n-1}{k-1}$$
\end{itemize}
\end{tcolorbox}

Soit $p \in \ ]0,1[$ et soit $X$ une variable aléatoire suivant la loi Binomiale $\mathcal{B} (5; p)$. \\

\textbf{1/} Calculer $\E (X)$. \\

\textbf{2/} Calculer $\E (X(X-1))$. \\

\textbf{3/} Calculer $\mathbb{V} (X)$. \\

\textbf{4/} Pour quelle valeur de $p$ la variance de $X$ est-elle maximale ? \\

\textbf{Bonus} (Loi uniforme) \\
\textit{À ne faire que si la totalité des exercices précédents ont été traités.}

\begin{tcolorbox}[colframe=black, colback=white, sharp corners, width=\textwidth, boxrule=0.5mm]
\textbf{Définition :}\\
Soit $n \in \n^*$. On dit que la variable aléatoire $X$ suit la $\textbf{loi uniforme}$ sur $\{1,2,...,n\}$ et on note $X \sim \mathcal{U} (\{1,2,...,n\})$ si $X$ prend ses valeurs dans $\{1,2,...,n\}$ et si pour tout $k \in \{1,2,...,n\}$, $\prob (X=k) = 1/n$. 
\end{tcolorbox}

On pose $n=19$. Soit $X \sim \mathcal{U} (\{1,2,...,n \})$ . \\

\textbf{1/} Calculer $\E (X)$. \\ 

\textbf{2/} On admet que $\sum_{k=1}^n k^2= \frac{n(n+1)(2n+1)}{6}$. Calculer $\mathbb{V}(X)$. \\

\end{document}