\documentclass[a4paper,12pt]{article}

% Packages
\usepackage[utf8]{inputenc}
\usepackage[T1]{fontenc}
\usepackage[french]{babel}
\usepackage{amsmath,amsfonts,amssymb}
\usepackage{graphicx}
\usepackage{geometry}
\usepackage{fancyhdr}
\usepackage{lastpage}
\usepackage{lipsum} % Pour générer du faux texte. À retirer dans le document final.
\usepackage{natbib}
\usepackage{bbm}
\usepackage{titlesec}
\usepackage{tcolorbox}

% Paramètres de la page
\geometry{top=2cm, bottom=2cm, left=2.5cm, right=2.5cm, footskip=1cm}
\pagestyle{fancy}
\fancyhf{}
\lhead{Outils Mathématiques : Interrogation 1}
\cfoot{\thepage} % Numéro de page au milieu du pied de page
\renewcommand{\headrulewidth}{2pt}
\renewcommand{\footrulewidth}{0pt} % Pas de ligne en bas
\setlength{\parindent}{0pt}

% Mise en page de la première page
\fancypagestyle{plain}{
  \fancyhf{} % Supprime les en-têtes et pieds par défaut
  \cfoot{\thepage} % Numéro de page au milieu
  \renewcommand{\headrulewidth}{0pt} % Pas de ligne d'en-tête
  \renewcommand{\footrulewidth}{0pt} % Pas de ligne en bas
}

% Titre du document
\title{Outils Mathématiques \\
Année 2024-2025}
\author{Interrogation 1}
\date{6 Mars 2025 \\
Durée : 1h}

% Macros
\newcommand{\E}{\mathbb{E}}
\newcommand{\prob}{\mathbb{P}}
\newcommand{\R}{\mathfrak{R}_0}
\newcommand{\Hun}{H^1(\Omega)}
\newcommand{\Ho}{H^1_0(\Omega)}
\newcommand{\vois}{\mathring{\mathcal{U}}}
\newcommand{\ind}{\mathbbm{1}}
\newcommand{\f}{\mathcal{F}}
\newcommand{\n}{\mathbb{N}}
\newcommand{\m}{\mathcal{M}}
\newcommand{\norm}{\mathcal{N}}
\newcommand{\sme}{\mathcal{E}}

% Chargement de hyperref avec les options personnalisées pour les cadres rouges
\usepackage[
    colorlinks=false,         % Pas de lien coloré
    linkbordercolor={1 0 0},  % Cadre rouge autour des liens
    pdfborder={0 0 1}         % Taille du cadre autour des liens
]{hyperref}

% Supprimer le décalage des sous-sections et masquer le numéro
\titleformat{\subsection}[hang]{\normalfont\bfseries}{}{0pt}{}

\begin{document}

\maketitle

\textbf{Attention : Lorsqu'un calcul est demandé, il est attendu que les étapes permettant d'aboutir au résultat soient détaillées. Plus généralement, toute réponse doit être justifiée.} \\


\textbf{Exercice 1} (Inversion matricielle) \\

\textbf{1/} Soient \( A = \begin{pmatrix}
2 & -2 \\
1 & -2 \\
\end{pmatrix} \) et \( B = \begin{pmatrix}
1 & 2 & -2 \\
0 & 1 & -2 \\
1 & 1 & 1
\end{pmatrix} \). \\

Inverser les matrices $A$ et $B$ en appliquant la méthode du pivot. \\

\textcolor{blue}{ On applique l'algorithme du pivot de Gauss. \\
Commençons par la matrice $A$.
\[ M_0=
\begin{pmatrix}
\boxed{2} & -2 & | & 1 & 0 \\
1 & -2 & | & 0 & 1
\end{pmatrix} 
\begin{array}{c} \\
L_2' = 2L_2 - L_1
\end{array}
\]
\[
\begin{aligned}
2L_2 &= \begin{pmatrix}
2 & -4 & | & 0 & 2
\end{pmatrix} \\
-L_1 &= \begin{pmatrix}
-2 & 2 & | & -1 & 0
\end{pmatrix} \\
L_2' &= \begin{pmatrix}
0 & -2 & | & -1 & 2
\end{pmatrix}
\end{aligned}
\]
\[ M_1=
\begin{pmatrix}
2 & -2 & | & 1 & 0 \\
0 & \boxed{-2} & | & -1 & 2
\end{pmatrix} 
\begin{array}{c} L_1' = L_1-L_2 
\\
\\
\end{array}
\]
\[
\begin{aligned}
L_1 &= \begin{pmatrix}
2 & -2 & | & 1 & 0 
\end{pmatrix} \\
-L_2 &= \begin{pmatrix}
0 & 2 & | & 1 & -2
\end{pmatrix} \\
L_1' &= \begin{pmatrix}
2 & 0 & | & 2 & -2 
\end{pmatrix}
\end{aligned}
\]
\[ M_2=
\begin{pmatrix}
2 & 0 & | & 2 & -2  \\
0 & -2 & | & -1 & 2
\end{pmatrix} 
\begin{array}{c} L_1' = \frac{1}{2} L_1 \\
L_2'= \frac{-1}{2} L_2
\end{array}
\]
\[ M_3=
\begin{pmatrix}
1 & 0 & | & 1 & -1  \\
0 & 1 & | & \frac{1}{2} & -1
\end{pmatrix}
\]
Donc $A$ est inversible et son inverse est \( A^{-1} = \begin{pmatrix}
1 & -1 \\
\frac{1}{2} & -1 \\
\end{pmatrix} \).
}

\textcolor{blue}{En ce qui concerne la matrice $B$, on a :
\[ M_0=
\begin{pmatrix}
\boxed{1} & 2 & -2 & | & 1 & 0 & 0 \\
0 & 1 & -2 & | & 0 & 1 & 0 \\
1 & 1 & 1 & | & 0 & 0 & 1
\end{pmatrix} 
\begin{array}{c} \\
\\
L_3' = L_3 - L_1
\end{array}
\]
\[
\begin{aligned}
L_3 &= \begin{pmatrix}
1 & 1 & 1 & | & 0 & 0 & 1
\end{pmatrix} \\
-L_1 &= \begin{pmatrix}
-1 & -2 & 2 & | & -1 & 0 & 0
\end{pmatrix} \\
L_3' &= \begin{pmatrix}
0 & -1 & 3 & | & -1 & 0 & 1
\end{pmatrix}
\end{aligned}
\]
\[ M_1=
\begin{pmatrix}
1 & 2 & -2 & | & 1 & 0 & 0 \\
0 & \boxed{1} & -2 & | & 0 & 1 & 0 \\
0 & -1 & 3 & | & -1 & 0 & 1
\end{pmatrix} 
\begin{array}{c} L_1' = L_1-2L_2 \\
 \\
L_3' = L_3 + L_2
\end{array}
\]
\[
\begin{aligned}
L_3 &= \begin{pmatrix}
0 & -1 & 3 & | & -1 & 0 & 1
\end{pmatrix} \\
L_2 &= \begin{pmatrix}
0 & 1 & -2 & | & 0 & 1 & 0
\end{pmatrix} \\
L_3' &= \begin{pmatrix}
0 & 0 & 1 & | & -1 & 1 & 1
\end{pmatrix}
\end{aligned}
\]
\[
\begin{aligned}
L_1 &= \begin{pmatrix}
1 & 2 & -2 & | & 1 & 0 & 0
\end{pmatrix} \\
-2L_2 &= \begin{pmatrix}
0 & -2 & 4 & | & 0 & -2 & 0
\end{pmatrix} \\
L_1' &= \begin{pmatrix}
1 & 0 & 2 & | & 1 & -2 & 0
\end{pmatrix}
\end{aligned}
\]
\[ M_2 = 
\begin{pmatrix}
1 & 0 & 2 & | & 1 & -2 & 0 \\
0 & 1 & -2 & | & 0 & 1 & 0 \\
0 & 0 & \boxed{1} & | & -1 & 1 & 1
\end{pmatrix} 
\begin{array}{c} L_1' = L_1-2L_3 \\
L_2' = L_2+2L_3 \\ \\
\end{array}
\]
\[
\begin{aligned}
L_1 &= \begin{pmatrix}
1 & 0 & 2 & | & 1 & -2 & 0
\end{pmatrix} \\
-2L_3 &= \begin{pmatrix}
0 & 0 & -2 & | & 2 & -2 & -2
\end{pmatrix} \\
L_1' &= \begin{pmatrix}
1 & 0 & 0 & | & 3 & -4 & -2
\end{pmatrix}
\end{aligned}
\]
\[
\begin{aligned}
L_2 &= \begin{pmatrix}
0 & 1 & -2 & | & 0 & 1 & 0
\end{pmatrix} \\
2L_3 &= \begin{pmatrix}
0 & 0 & 2 & | & -2 & 2 & 2
\end{pmatrix} \\
L_2' &= \begin{pmatrix}
0 & 1 & 0 & | & -2 & 3 & 2
\end{pmatrix}
\end{aligned}
\]
\[ M_3 = 
\begin{pmatrix}
1 & 0 & 0 & | & 3 & -4 & -2 \\
0 & 1 & 0 & | & -2 & 3 & 2 \\
0 & 0 & 1 & | & -1 & 1 & 1
\end{pmatrix} 
\]
Donc $B$ est inversible et son inverse est \( B^{-1} = \begin{pmatrix}
3 & -4 & -2 \\
-2 & 3 & 2 \\
-1 & 1 & 1
\end{pmatrix} \).
}

\textbf{2/} Soit \( C = \begin{pmatrix}
0 & 1 & -1 \\
-1 & 2 & -1 \\
1 & -1 & 2
\end{pmatrix} \).  \\

\textbf{a)} Calculer $C^2$ puis $C^2-3C$. \\

\textcolor{blue}{On a 
\( C^2 = \begin{pmatrix}
-2 & 3 & -3 \\
-3 & 4 & -3 \\
3 & -3 & 4
\end{pmatrix} \) et $C^2-3C=-2I_3$.} \\

\textbf{b)} En déduire que $C$ est inversible et donner l'expression de son inverse $C^{-1}$. \\

\textcolor{blue}{Ainsi, on a $\frac{-1}{2} (C-3I_3)C = I_3$. \\
Donc $C$ est inversible et $C^{-1} = \frac{-1}{2} (C-3I_3) =$ \( \begin{pmatrix}
\frac{3}{2} & \frac{-1}{2} & \frac{1}{2} \\
\frac{1}{2} & \frac{1}{2} & \frac{1}{2} \\
\frac{-1}{2} & \frac{1}{2} & \frac{1}{2}
\end{pmatrix} \)} \\

\textbf{Exercice 2} (Diagonalisation) \\

Soit \( A = \begin{pmatrix}
3 & 0 & 0 \\
2 & 2 & 0 \\
4 & 0 & 1
\end{pmatrix} \).\\
\textbf{1/} Calculer le polynôme caractéristique de la matrice $A$ en le laissant sous forme factorisée. En déduire l'ensemble des valeurs propres de la matrice $A$. \\

\textcolor{blue}{Le polynôme caractéristique de $A$ est $P_A(X)=(X-1)(X-2)(X-3)$. L'ensemble des valeurs propres de $A$ est donc $\{1,2,3\}$.} \\

\textbf{2/} Trouver une base de chacun des sous-espaces propres de la matrice $A$. \\

\textcolor{blue}{
\underline{Espace propre associé à la valeur propre $\lambda = 3$.} \\ \\
Soit $X =
\begin{pmatrix}
x \\
y \\
z
\end{pmatrix}
$. $AX = 3X$ si et seulement si :
\[
\begin{cases}
3x = 3x \\
2x + 2y = 3y \\
4x + z = 3z
\end{cases}
\quad \text{ssi} \quad
\begin{cases}
2x + 2y = 3y \\
4x + z = 3z
\end{cases}
\quad \text{ssi} \quad
\begin{cases}
y = 2x \\
2z = 4x
\end{cases}
\quad \text{ssi} \quad
\begin{cases}
y = 2x \\
z = 2x
\end{cases}
\]
Ainsi, $AX = 3X$ si et seulement s'il existe $x \in \mathbb{R}$ tel que $X =
\begin{pmatrix}
x \\
2x \\
2x
\end{pmatrix} = x e_1
$ avec $e_1 =
\begin{pmatrix}
1 \\
2 \\
2
\end{pmatrix}$.
Donc $(e_1)$ est base de $E_3(A)$.}
\\
\textcolor{blue}{
\underline{Espace propre associé à la valeur propre $\lambda = 2$.} \\ \\
Soit $X =
\begin{pmatrix}
x \\
y \\
z
\end{pmatrix}
$. $AX = 2X$ si et seulement si :
\[
\begin{cases}
3x = 2x \\
2x + 2y = 2y \\
4x + z = 2z
\end{cases}
\quad \text{ssi} \quad
\begin{cases}
x = 0 \\
2y = 2y \\
z = 2z
\end{cases}
\quad \text{ssi} \quad
\begin{cases}
x = 0 \\
z = 0
\end{cases}
\]
Ainsi, $AX = 2X$ si et seulement s'il existe $y \in \mathbb{R}$ tel que $X =
\begin{pmatrix}
0 \\
y \\
0
\end{pmatrix} = y e_2
$ avec $e_2 =
\begin{pmatrix}
0 \\
1 \\
0
\end{pmatrix}$.
Donc $(e_2)$ est base de $E_2(A)$.} \\

\textcolor{blue}{ 
\underline{Espace propre associé à la valeur propre $\lambda = 1$.} \\ \\
Soit $X =
\begin{pmatrix}
x \\
y \\
z
\end{pmatrix}
$. $AX=X$ si et seulement si :
\[
\begin{cases}
3x = x \\
2x + 2y = y \\
4x + z = z
\end{cases}
\quad \text{ssi} \quad
\begin{cases}
x = 0 \\
2y = y \\
z = z
\end{cases}
\quad \text{ssi} \quad
\begin{cases}
x = 0 \\
y = 0
\end{cases}
\]
Ainsi, $AX=X$ si et seulement s'il existe $z \in \mathbb{R}$ tel que $X =
\begin{pmatrix}
0 \\
0 \\
z
\end{pmatrix} = z e_3
$ avec $e_3 =
\begin{pmatrix}
0 \\
0 \\
1
\end{pmatrix}$. $(e_3)$ est base de $E_1(A)$.
} \\

\textbf{3/} La matrice $A$ est-elle diagonalisable ? Justifier. \\

\textcolor{blue}{La famille $\boxed{\underline{b}'=(e_1,e_2,e_3)}$ est une base de vecteurs propres de $A$. La matrice $A$ est donc diagonalisable.} \\

\textbf{Exercice 3} (Loi binomiale) \\

\begin{tcolorbox}[colframe=black, colback=white, sharp corners, width=\textwidth, boxrule=0.5mm]
\textbf{Rappel :}\\
\begin{itemize}
    \item On dit que la variable aléatoire $X$ suit la \textbf{loi binomiale} de paramètres $n \in \n^*$, $p \in \ ]0,1[$ et on note $X \sim \mathcal{B} (n,p)$ si $X$ prend ses valeurs dans $\{0,1,...,n\}$ et si pour tout $k \in \{0,1,...,n\}$, $\prob (X=k) = \binom{n}{k} p^k (1-p)^{n-k}$. \\

\item On rappelle la \textbf{formule du capitaine}. Pour $1 \leq k \leq n$, on a :
$$k \binom{n}{k} = n \binom{n-1}{k-1}$$
\end{itemize}
\end{tcolorbox}

Soit $p \in \ ]0,1[$ et soit $X$ une variable aléatoire suivant la loi Binomiale $\mathcal{B} (5; p)$. \\

\textbf{1/} Calculer $\E (X)$. \\
\textcolor{blue}{
On a :
\begin{align*}
\E(X) &=  \sum_{k=0}^n k \prob (X=k) \quad \text{formule de transfert} \\
&= \sum_{k=1}^n k \binom{n}{k} p^k (1-p)^{n-k} \quad \text{1er terme nul} \\
&= n \sum_{k=1}^n \binom{n-1}{k-1} p^k (1-p)^{n-k} \quad \text{formule du capitaine}  \\
&= n \sum_{l=0}^{n-1} \binom{n-1}{l} p^{l+1} (1-p)^{n-(l+1)} \quad \text{cdv } l=k-1 \\
&= np \sum_{l=0}^{n-1} \binom{n-1}{l} p^l (1-p)^{(n-1)-l} \\
&= np \left( p + (1-p) \right)^{n-1} \quad \text{binôme de Newton} \\
&= np \\
& = 5p
\end{align*}
}

\textbf{2/} Calculer $\E (X(X-1))$. \\

\textcolor{blue}{
Par ailleurs :
\begin{align*}
\E(X(X-1)) &= \sum_{k=0}^n k(k-1) \prob(X=k) \\
&= \sum_{k=2}^n k(k-1) \prob(X=k) \quad \text{on retire les termes nuls} \\
&= \sum_{k=2}^n k(k-1) \binom{n}{k} p^k (1-p)^{n-k}
\end{align*}
}

\textcolor{blue}{
D'après la formule du capitaine (que l'on applique deux fois), on a pour $2 \leq k \leq n$ :
\[
k(k-1) \binom{n}{k} = n(k-1) \binom{n-1}{k-1} = n(n-1) \binom{n-2}{k-2}
\]
}

\textcolor{blue}{
Si bien que :
\begin{align*}
\E(X(X-1)) &= n(n-1) \sum_{k=2}^n \binom{n-2}{k-2} p^{k}(1-p)^{n-k} \\
&= n(n-1) \sum_{l=0}^{n-2} \binom{n-2}{l} p^{l+2} (1-p)^{n-(l+2)} \quad \text{cdv } l=k-2 \\
&= p^2 n(n-1) \sum_{l=0}^{n-2} \binom{n-2}{l} p^{l} (1-p)^{(n-2)-l} \\
&= p^2 n(n-1) \left( p + (1-p) \right)^{n-2} \quad \text{binôme de Newton} \\
&= p^2 n(n-1) \\
& = 20p^2
\end{align*}
}

\textbf{3/} Calculer $\mathbb{V} (X)$. \\

\textcolor{blue}{
En écrivant $X^2 = X(X-1) + X$, on a :
\begin{align*}
\mathbb{V}(X) &= \E(X^2) - \E(X)^2 \\
&= \E(X(X-1) + X) - \E(X)^2 \\
&= \E(X(X-1)) + \E(X) - \E(X)^2 \quad \text{linéarité de l'espérance} \\
&= p^2 n(n-1) + np - n^2 p^2 \\
&= n^2 p^2 - np^2 + np - n^2 p^2 \\
&= -np^2 + np \\
&= np(1-p) \\
&\boxed{= 5p(1-p)}
\end{align*}
}

\textbf{4/} Pour quelle valeur de $p$ la variance de $X$ est-elle maximale ? \\

\textcolor{blue}{La variance de $X$ s'écrit $\mathbb{V} (X) = -5p^2 +5p$. Il s'agit d'une fonction polynomiale en $p$ de degré $2$, qui atteint son maximum sur $]0,1[$ en $p=\frac{1}{2}$.} \\

\textbf{Bonus} (Loi uniforme) \\
\textit{À ne faire que si la totalité des exercices précédents ont été traités.}

\begin{tcolorbox}[colframe=black, colback=white, sharp corners, width=\textwidth, boxrule=0.5mm]
\textbf{Définition :}\\
Soit $n \in \n^*$. On dit que la variable aléatoire $X$ suit la $\textbf{loi uniforme}$ sur $\{1,2,...,n\}$ et on note $X \sim \mathcal{U} (\{1,2,...,n\})$ si $X$ prend ses valeurs dans $\{1,2,...,n\}$ et si pour tout $k \in \{1,2,...,n\}$, $\prob (X=k) = 1/n$. 
\end{tcolorbox}

On pose $n=19$. Soit $X \sim \mathcal{U} (\{1,2,...,n \})$ . \\

\textbf{1/} Calculer $\E (X)$. \\ 

\textcolor{blue}{On a :
\begin{align*}
\E (X) & = \sum_{k=1}^n k \prob (X=k) \\
& = \sum_{k=1}^n \frac{k}{n} \\
& = \frac{1}{n} \frac{n(n+1)}{2} \ \ \text{somme des premiers termes d'une suite arithmétique} \\
& = \frac{n+1}{2} \\
& = 10
\end{align*}}

\textbf{2/} On admet que $\sum_{k=1}^n k^2= \frac{n(n+1)(2n+1)}{6}$. Calculer $\mathbb{V}(X)$. \\

\textcolor{blue}{
On a d'abord :
\begin{align*}
\E(X^2) & = \sum_{k=1}^n k^2 \prob (X=k) \\
& = \sum_{k=1}^n \frac{k^2}{n} \\
& = \frac{1}{n} \frac{n(n+1)(2n+1)}{6} \ \ \text{indication de l'énoncé} \\
& = \frac{(n+1)(2n+1)}{6}
\end{align*}
Puis :
\begin{align*}
\mathbb{V} (X) & = \E (X^2) - \E(X)^2 \\
& = \frac{(n+1)(2n+1)}{6} - \frac{(n+1)^2}{4} \ \ \text{par ce qui précède} \\
& = \frac{n^2-1}{12} \ \ \text{en mettant au même dénominateur} \\
& = 30
\end{align*}}

\end{document}